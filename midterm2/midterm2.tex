\def\thecourse{18.466}
\def\thestudent{Zhilei Xu (929552018)}
\def\theprob{Midterm 2}
\documentclass[11pt]{article}
\usepackage{amsmath, graphicx, amsfonts}
\usepackage{fancyhdr}
\usepackage{lastpage}
\pagestyle{fancy}
\fancyhf{}
\fancyhf[HLEO]{\thestudent{}}
\fancyhf[HCEO]{\thecourse{} - Problem Set \theprob{}}
\fancyhf[HREO]{Page \thepage\ of \pageref{LastPage}}
\renewcommand\headrulewidth{0.4pt}
\newcommand{\argmax}{\mathrm{argmax}}
\newcommand\erf{\mathrm{erf}}
\newcommand\sgn{\mathrm{sgn}}
\newcommand{\widesim}[2][1.5]{
  \mathrel{\overset{#2}{\scalebox{#1}[1]{$\sim$}}}
}
\newcommand\eqnlabel[1]{\label{eqn:#1}}
\newcommand\eqnref[1]{(\ref{eqn:#1})}
\newcommand{\Beta}{\mathcal{B}}
\newcommand{\ProbS}{\iftrue}
\newcommand{\ProbE}{\fi}

\usepackage{titlesec, pgfplots, subcaption}
\titleformat{\section}[runin]{\Large\bfseries}{}{}{}
\titleformat{\subsection}[runin]{\normalfont\large\bfseries}{}{}{}
\begin{document}
\section*{1}
\subsection*{(a)}

\section*{2}
\subsection*{(a)}
Figures \ref{fig:linex}(\subref{fig:linex:0.2} - \subref{fig:linex:1}) are the plots for.
\begin{figure}
\centering
\foreach \i in {0.2, 0.5, 1} {
\subcaptionbox{c=\i \label{fig:linex:\i}}{
\begin{tikzpicture}
  \begin{axis}[
    xlabel={$a-\theta$},
    %ylabel={$l(\theta, a), c=\i$},
    xmin = -4.5,
    xmax = 3.5,
    xtick = {-4, -2, 0, 1, ..., 3},
    width = 0.333\textwidth
  ] 
  \addplot +[mark=none] {e^(\i * x) - \i*x -1}
  %[yshift=8pt]
  %  node[pos=0.5] {$e^{\i(a-\theta)}-\i(a-\theta)-1$}
  ;
 
  \end{axis}
\end{tikzpicture}
}
}
\caption{plots for LINEX $l(\theta, a)$ under different $c$}
\label{fig:linex}
\end{figure}


\end{document}
